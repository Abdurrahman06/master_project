\documentclass[11pt]{article} 

% ***********************************************************
% ******************* PHYSICS HEADER ************************
% ***********************************************************
% Version 2
%\usepackage{MnSymbol}
\newtheorem{guess}{Hypothesis}
%\usepackage[b5paper]{geometry}
\usepackage{subcaption}

\usepackage{mathtools}   % loads »amsmath«
\usepackage{empheq}


\usepackage{epsfig}
%\usepackage{mdframed}
\usepackage{tikz}
\usetikzlibrary{scopes}
\usepackage{amsmath}
\usepackage{bm}
\usepackage{amsthm } % Theorem Formatting
\usepackage{amssymb}	% Math symbols such as \mathbb
\usepackage{listings}

\usepackage{graphicx}

%\usepackage{hyperref}
\usepackage{multicol} % Allows for multiple columns

\usepackage{fullpage, float, subfig, verbatim,amsmath, mathrsfs}
\usepackage{fancyhdr}
\usepackage{natbib}
\usepackage{appendix}
\usepackage{alltt}
%%%%
\usepackage{breqn}
\usepackage{fixltx2e}

\sloppy
\definecolor{lightgray}{gray}{0.5}
%%%%
\fancyhf{}
\headsep=20pt

% algorithm
\usepackage{algorithm}
\usepackage{algorithmic}


\usepackage[glenn]{fncychap}
\usepackage{sectsty}
\allsectionsfont{\sffamily
%\mdseries
\upshape } % (See the fntguide.pdf for font help)


%Change the chapte layout
\usepackage{titlesec}
\titlespacing*{\chapter}{0pt}{-50pt}{20pt}
\titleformat{\chapter}[display]{\ttfamily \LARGE\bfseries}{\chaptertitlename\ \thechapter}{22pt}{\Huge}

% Fancy MATLAB-kode:
\usepackage{textcomp}
\definecolor{lbcolor}{rgb}{0.95,0.95,0.95}
\lstset{
%	backgroundcolor=\color{lbcolor},
	tabsize=4,
	rulecolor=,
	language=matlab,
        basicstyle=\scriptsize,
        upquote=true,
        aboveskip={1.5\baselineskip},
        columns=fixed,
        showstringspaces=false,
        extendedchars=true,
        breaklines=true,
        prebreak = \raisebox{0ex}[0ex][0ex]{\ensuremath{\hookleftarrow}},
        frame=single,
        showtabs=true,
        showspaces=false,
        showstringspaces=false,
        identifierstyle=\ttfamily,
        keywordstyle=\color[rgb]{0,0,1},
        commentstyle=\color[rgb]{0.133,0.545,0.133},
        stringstyle=\color[rgb]{0.627,0.126,0.941},
        numbers=left, 
        numberstyle=\tiny, 
        stepnumber=1, 
        numbersep=5pt
}

% equations
\definecolor{myeqcolor}{gray}{0.9}
\newcommand*\mygraybox[1]{%
\colorbox{myeqcolor}{\hspace{1em}#1\hspace{1em}}}

% more equation highlightin
%\usepackage{framed}
\usepackage[framemethod=TikZ]{mdframed}
\usepackage{xcolor}
\surroundwithmdframed[
    hidealllines=true,
    backgroundcolor=black!20,
    skipbelow=\baselineskip,
    skipabove=\baselineskip
]{equation}
\global\mdfdefinestyle{graybox}{%
outerlinewidth=0pt,innerlinewidth=0pt,
outerlinecolor=gray,roundcorner=0pt,
backgroundcolor=gray!27
}

\usepackage{longtable}

%\usepackage[svgnames]{xcolor} % Required to specify font color

\newcommand*{\plogo}{\fbox{$\mathcal{PL}$}} % Generic publisher logo

%----------------------------------------------------------------------------------------
%	TITLE PAGE
%----------------------------------------------------------------------------------------

\newcommand*{\titleAT}{\begingroup % Create the command for including the title page in the document
\newlength{\drop} % Command for generating a specific amount of whitespace
\drop=0.1\textheight % Define the command as 10% of the total text height

\rule{\textwidth}{1pt}\par % Thick horizontal line
\vspace{2pt}\vspace{-\baselineskip} % Whitespace between lines
\rule{\textwidth}{0.4pt}\par % Thin horizontal line

\vspace{\drop} % Whitespace between the top lines and title
\centering % Center all text
%\textcolor{Red}{ % Red font color
{\Huge A TITLE}

\vspace{0.25\drop} % Whitespace between the title and short horizontal line
\rule{0.3\textwidth}{0.4pt}\par % Short horizontal line under the title
\vspace{\drop} % Whitespace between the thin horizontal line and the author name
by \\
\vspace{0.25\drop} % Whitespace between the thin horizontal line and the author name
{\Large \textsc{Simen Andresen}}\par % Author name
\vfill % Whitespace between the author name and publisher text

NTNU
\rule{\textwidth}{0.4pt}\par % Thin horizontal line
\vspace{2pt}\vspace{-\baselineskip} % Whitespace between lines
\rule{\textwidth}{1pt}\par % Thick horizontal line

\endgroup}



\linespread{1.5}




% algorithms
\floatstyle{plain}
\newfloat{myalgo}{tbhp}{mya}

\newenvironment{Algorithm}[2][tbh]%
{\begin{myalgo}[#1]
\centering
\begin{minipage}{#2}
\begin{algorithm}[H]}%
{\end{algorithm}
\end{minipage}
\end{myalgo}}

% then use the following
		%\begin{Algorithm}[t]
		%\caption{Does work, though no nice solution.}
		%\end{Algorithm}



\usepackage{booktabs}
\usepackage[]{geometry}
%\usepackage[b5paper]{geometry}
\textwidth = 15cm
